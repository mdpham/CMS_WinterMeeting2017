\documentclass[11pt, oneside]{article}   	% use "amsart" instead of "article" for AMSLaTeX format
\usepackage{geometry}                		% See geometry.pdf to learn the layout options. There are lots.
\geometry{letterpaper}                   		% ... or a4paper or a5paper or ... 
%\geometry{landscape}                		% Activate for rotated page geometry
%\usepackage[parfill]{parskip}    		% Activate to begin paragraphs with an empty line rather than an indent
\usepackage{graphicx}				% Use pdf, png, jpg, or eps§ with pdflatex; use eps in DVI mode
								% TeX will automatically convert eps --> pdf in pdflatex		
\usepackage{amssymb}

%SetFonts

%SetFonts


\title{CMS Winter Meeting 2017: The Inverse Fractal Problem For Audio Compression}
\author{Martin Pham\\ University of Waterloo \\ Department of Applied Math}
%\date{}							% Activate to display a given date or no date

\begin{document}
\maketitle

In this talk we present the iterated function system (IFS) approach for generating fractals -objects that exhibit self-similarity and scale-free complexity- and its application in data compression.
The local IFS method is introduced for image compression in order to motivate an extension to one-dimensional audio signals.
A frequency domain analysis of preliminary compression results is then discussed, with emphasis on the comparison between images and audio for fractal applications.
Finally, earlier work and further directions towards fractal-wavelet methods are considered.



% MAKE IT SIMPLE AND TO GENERAL AUDIENCE. DON'T DESCRIBE THE ALGORITHM IN DETAIL
%An investigation into the effects of applying a local iterated function system approach to 1D signal compression is presented.
%Also called the collage approach, the signal is subdivided into two separate sets of frames: domain blocks and range blocks, where the length of a domain block is twice that of a range block.
%Each range block is approximated by a domain block by an appropriate contractive affine transformation.
%The compression procedure requires a global search over all domain blocks for each range block in order to determine the domain block/affine transformation pair that minimizes the approximation error.
%Together the affine transformations and associated domain blocks define a fractal transform whose fixed point well approximates the original signal.
%Thus an approximation of the signal may be recovered by repeated application of the transform from any initial signal.
%The drawbacks for applying fractal methods in the spatial domain are explored by considering the frequency artefacts introduced by the compression.
%Further directions toward a fractal-wavelet local iterated function system approach are discussed.





\end{document}  